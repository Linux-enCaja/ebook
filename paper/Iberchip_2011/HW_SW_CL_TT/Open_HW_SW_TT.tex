\documentclass[conference,a4paper]{IEEEtran}
\usepackage[latin1]{inputenc}
\usepackage{enumerate}
\usepackage{cite}      
\usepackage{graphicx}  
\usepackage{tikz}
\usepackage{psfrag}    
\usepackage{subfigure} 
\usepackage{multicol}
\usepackage{multirow}
\usepackage{colortbl}
\usepackage{floatflt}
\usepackage{amsmath}   % From the American Mathematical Society
\usepackage[mathscr]{euscript}%Para letra en modo matem�ico
\usepackage{pst-all,pstricks,graphics,graphpap,amsmath,amssymb,latexsym,array}

\newpsobject{grilla}{psgrid}{subgriddiv=1,griddots=10,gridlabels=6pt}
\interdisplaylinepenalty=2500
\usepackage{array}

\hyphenation{semi-conduc-tor }

\DeclareMathOperator{\sen}{sen}

\begin{document}
\bibliographystyle{unsrt}

%%%%%%%%%%%%%%%%%%%%%%%%%%%%%%%%%%%%%%%%%%%%%%
%                 TITLE                      %
%%%%%%%%%%%%%%%%%%%%%%%%%%%%%%%%%%%%%%%%%%%%%%

% \title{Software Libre y Hardware Copyleft Herramientas para la Transferencia Tecnol�gica en el Dise�o e Implementaci�n de Sistemas Embebidos}


\title{Metodolog�a Para la Transferencia Tecnol�gica en la Industria Electr�nica Basada en Software Libre y Hardware Copyleft}


\author{
\IEEEauthorblockN{Carlos I. Camargo Bare�o}
\IEEEauthorblockA{Universidad Nacional de Colombia,
Email: cicamargoba@unal.edu.co}
}

\maketitle



%%%%%%%%%%%%%%%%%%%%%%%%%%%%%%%%%%%%%%%%%%%%%%
%                 ABSTRACT                   %
%%%%%%%%%%%%%%%%%%%%%%%%%%%%%%%%%%%%%%%%%%%%%%
\begin{abstract}
Los canales tradicionales para la transferencia tecnol�gica en el �rea del dise�o de sistemas embebidos no han sido exitosos en los pa�ses en v�a de desarrollo donde la plataforma tecnol�gica no est� lo suficientemente desarrollada para absorber esta nueva tecnolog�a y conocimientos, esto se debe en gran parte a que las empresas de base tecnol�gica son muy peque�as, con bajos niveles de producci�n y utilizan metodolog�as de dise�o desactualizadas; por otro lado, la dependencia hacia los productos asi�ticos hace muy dif�cil que estas peque�as empresas puedan crecer sin la protecci�n del estado. As� mismo, exite una sobre-oferta de profesionales afines con la industria electr�nica, una gran parte de ellos provienen de entidades poco consolidadas. La uni�n de estos factores genera una tasa de desempleo muy alta, salarios bajos, aumento de la dependencia a los productos extranjeros y una desconfianza hacia los productos generados localmente; lo que afecta de forma considerable el n�mero de estudiantes que ingresan a los programas de formaci�n relacionados con la electr�nica, llegando hasta el punto del cierre de programas acreditados.

Este art�culo presenta una metodolog�a para la transferencia tecnol�gica en el dise�o de sistemas embebidos desarrollada en el Departamento de ingenier�a el�ctrica y electr�nica de la Universidad Nacional de Colombia; esta metodolog�a tiene como pilares el conocimiento como bien com�n, el movimiento de software libre y un concepto nuevo desarrollado en conjunto con un grupo de desarrolladores hardware y software: el \textit{hardware copyleft}.

\end{abstract}


\begin{keywords}
Sistemas Embebidos, educaci�n en ingenier�a, hardware copyleft.
\end{keywords}


\IEEEpeerreviewmaketitle

  \input intro

\section{BIBLIOGRAF�A}
\bibliography{./biblio.bib}

\end{document}