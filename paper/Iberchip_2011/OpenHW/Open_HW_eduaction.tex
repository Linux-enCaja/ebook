\documentclass[conference,a4paper]{IEEEtran}
\usepackage[latin1]{inputenc}
% \usepackage{enumerate}
\usepackage{cite}      
\usepackage{graphicx}  
\usepackage{tikz}
% \usepackage{verbatim}
% \usepackage[floats,active,tightpage]{preview}
% \usetikzlibrary{backgrounds,shapes,arrows,positioning,calc,snakes,fit}
% \usepgflibrary{decorations.markings}

\usepackage{psfrag}    
\usepackage{subfigure} 
\usepackage{multicol}
\usepackage{multirow}
\usepackage{colortbl}
\usepackage{floatflt}
\usepackage{amsmath}   % From the American Mathematical Society
\usepackage[mathscr]{euscript}%Para letra en modo matem�ico
\usepackage{pst-all,pstricks,graphics,graphpap,amsmath,amssymb,latexsym,array}


\newpsobject{grilla}{psgrid}{subgriddiv=1,griddots=10,gridlabels=6pt}
\interdisplaylinepenalty=2500
\usepackage{array}

\hyphenation{semi-conduc-tor}

\DeclareMathOperator{\sen}{sen}

\begin{document}
\bibliographystyle{unsrt}

%%%%%%%%%%%%%%%%%%%%%%%%%%%%%%%%%%%%%%%%%%%%%%
%                 TITLE                      %
%%%%%%%%%%%%%%%%%%%%%%%%%%%%%%%%%%%%%%%%%%%%%%

\title{SIE: Plataforma Hardware \textit{copyleft}  para la Ense�anza de Sistemas Digitales}

\author{Carlos Iv�n Camargo -- Universidad Nacional de Colombia\\\ cicamargoba@unal.edu.co}
  
% \markboth{CAMARGO: SIE: Plataforma Hardware \textit{copyleft}  para la Ense�anza de Sistemas Digitales}
% {Iberchip 2011}

\maketitle
\IEEEpeerreviewmaketitle


%%%%%%%%%%%%%%%%%%%%%%%%%%%%%%%%%%%%%%%%%%%%%%
%                 ABSTRACT                   %
%%%%%%%%%%%%%%%%%%%%%%%%%%%%%%%%%%%%%%%%%%%%%%
\begin{abstract}
El gran avance de las t�cnicas de fabricaci�n de Circuitos Integrados ha permitido que los sistemas embebidos sean parte fundamental de nuestra vida, a�n sin darnos cuenta diariamente interactuamos con decenas de ellos; esto unido a la disponibilidad de herramientas software de desarrollo gratuitas (compiladores, simuladores, librer�as, Sistemas Operativos, herramientas CAD) abre grandes posibilidades comerciales para paises en v�a de desarrollo ya que no son necesarias grandes inversiones de capital para la concepci�n, dise�o, y fabricaci�n de estos sistemas. 

Los dispositivos \textit{hardware copyleft} suministran la informaci�n necesaria para reproducir, modificar y construir el dispositivo f�sico, son el complemento ideal del software libre y del c�digo abierto; la uni�n software libre y hardware copyleft constituye una herramienta muy poderosa para la transferencia tecnol�gica y de conocimientos en el dise�o de sistemas digitales ya que pueden ser utilizadas como referencia para nuevos productos o como material de estudio. En este art�culo se presentan las caracter�sticas de una plataforma \textit{hardware copyleft} y su aplicaci�n en la ense�anza del dise�o digital.

\end{abstract}

\begin{keywords}
Educaci�n en Ingenier�a, Sistemas Embebidos, Hardware copyleft, Co-dise�o HW/SW.
\end{keywords}

  \input intro

\bibliography{./biblio.bib}

\end{document}