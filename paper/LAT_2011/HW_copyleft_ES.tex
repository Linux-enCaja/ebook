\documentclass[conference,a4paper]{IEEEtran}
\usepackage[latin1]{inputenc}
\usepackage{enumerate}
\usepackage{cite}      
\usepackage{graphicx}  
\usepackage{tikz}
\usepackage{psfrag}    
\usepackage{subfigure} 
\usepackage{multicol}
\usepackage{multirow}
\usepackage{colortbl}
\usepackage{floatflt}
\usepackage{amsmath}   % From the American Mathematical Society
\usepackage[mathscr]{euscript}%Para letra en modo matem�ico
\usepackage{pst-all,pstricks,graphics,graphpap,amsmath,amssymb,latexsym,array}

\newpsobject{grilla}{psgrid}{subgriddiv=1,griddots=10,gridlabels=6pt}
\interdisplaylinepenalty=2500
\usepackage{array}

\hyphenation{semi-conduc-tor }

\DeclareMathOperator{\sen}{sen}

\begin{document}
\bibliographystyle{unsrt}

%%%%%%%%%%%%%%%%%%%%%%%%%%%%%%%%%%%%%%%%%%%%%%
%                 TITLE                      %
%%%%%%%%%%%%%%%%%%%%%%%%%%%%%%%%%%%%%%%%%%%%%%
\title{El papel del Hardware \textit{copyleft} en la Ense�anza de Sistemas Embebidos}

\author{

\IEEEauthorblockN{Carlos I. Camargo Bare�o}
\IEEEauthorblockA{Universidad Nacional de Colombia,\\
Email: cicamargoba@unal.edu.co}
}
\maketitle

%%%%%%%%%%%%%%%%%%%%%%%%%%%%%%%%%%%%%%%%%%%%%%
%                 ABSTRACT                   %
%%%%%%%%%%%%%%%%%%%%%%%%%%%%%%%%%%%%%%%%%%%%%%
\begin{abstract}
El gran avance de las t�cnicas de fabricaci�n de Circuitos Integrados ha permitido que los sistemas embebidos sean parte fundamental de nuestras vidas, a�n sin darnos cuenta diariamente interactuamos con decenas de ellos. Esto unido a la disponibilidad de herramientas software de desarrollo gratuitas abre grandes posibilidades comerciales para paises en v�a de desarrollo ya que no son necesarias grandes inversiones de capital para la concepci�n, dise�o, y fabricaci�n de estos sistemas. Sin embargo, en la actualidad muy pocas universidades ofrecen cursos que permitan crear las habilidades necesarias para la realizaci�n de un producto comercializable, lo que se traduce en un abandono de la producci�n local y el aumento de la dependencia con la industria manufacturera asi�tica. por otro lado, las herramientas utilizadas en la actualidad (tanto SW como HW) proporcionan un nivel de abstracci�n relativamente alto impidiendo que el estudiante entienda el funcionamiento global de un sistema digital, lo que le impide generar habilidades (especificamente las relacionadas con la concepci�n e implementaci�n) necesarias para realizar el proceso completo. En este art�culo se presenta una metodolog�a para la ense�anza de dise�o de sistemas embebidos utilizando herramientas hardware y software abiertas que ayuda a resolver los problemas mencionados anteriormente.
\end{abstract}

\begin{keywords}
Sistemas Embebidos, educaci�n en ingenier�a, hardware copyleft.
\end{keywords}

\IEEEpeerreviewmaketitle

\input intro
 
\bibliography{./biblio}

\end{document}
