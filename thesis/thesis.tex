\documentclass{book}

\usepackage{fontenc}
\usepackage[latin1]{inputenc}
\usepackage[spanish]{babel}

\usepackage{tikz}
\usetikzlibrary{arrows,shapes,snakes,automata,backgrounds,petri}

\usepackage{times}
\usepackage{multicol}
\usepackage{array}
\usepackage{multirow}
\usepackage{pdfpages}
\usepackage{listings}
\lstset{language=C} 
% \lstset{ basicstyle=\small, keywordstyle=\color{blue}\bfseries, identifierstyle=, commentstyle=\color{gray}, showstringspaces=false, keywordstyle=\color{blue}, morekeywords={one,three,five}}
\lstset{backgroundcolor=\color{gray},frame=single,emph={EMPTY},emphstyle=\color{white}, showstringspaces=false,commentstyle=\itshape, morekeywords={printf}, tabsize=2, fontadjust=true,
basicstyle=\footnotesize}
%\lstset{framexleftmargin=5mm, frame=shadowbox, rulesepcolor=\color{black}, backgroundcolor=\color[gray]{.9}}
\lstset{backgroundcolor=\color[gray]{.8}}

\usepackage{float}
\usepackage{rotating}
\usepackage{subfigure}




\usepackage{pstricks}


\definecolor{code-color}{cmyk}{0, 0, 0, 0.1}

\usepackage{amsmath,amssymb,latexsym,array, bm, times}

\pagestyle{myheadings}
\markboth{Sistemas Embebidos}{Carlos Iv�n Camargo Bare�o}

\usepackage{longtable}
\parindent 1cm
\parskip 0.2cm
\topmargin 0.2cm
\oddsidemargin 1cm
\evensidemargin 0.5cm
\textwidth 15cm
\textheight 21cm


\begin{document}
\bibliographystyle{unsrt}

\tableofcontents

\parindent=1cm            
%    \input title
%    \input intro
%     \input review
%    \input embedded
%     \input commons
    \input education

\bibliography{biblio}

\end{document} 



Such evidence stares at us from the performance of several Asian countries in the past few decades. These countries seem to understand that job creation must be the No. 1 objective of state economic policy. The government plays a strategic role in setting the priorities and arraying the forces and organization necessary to achieve this goal. The rapid development of the Asian economies provides numerous illustrations. In a thorough study of the industrial development of East Asia, Robert Wade of the London School of Economics found that these economies turned in precedent-shattering economic performances over the '70s and '80s in large part because of the effective involvement of the government in targeting the growth of manufacturing industries.




