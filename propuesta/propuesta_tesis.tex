\documentclass{book}

\usepackage{fontenc}
\usepackage[latin1]{inputenc}
\usepackage[spanish]{babel}
\usepackage[usenames,dvipsnames]{color}

\usepackage{tikz}
\usepackage{colortbl}

\usetikzlibrary{arrows,shapes,snakes,automata,backgrounds,petri}

\usepackage{times}
\usepackage{multicol}
\usepackage{multirow}
\usepackage{pdfpages}
\usepackage{listings}
\lstset{language=C} 
% \lstset{ basicstyle=\small,		 keywordstyle=\color{blue}\bfseries, identifierstyle=, commentstyle=\color{gray}, showstringspaces=false, keywordstyle=\color{blue}, morekeywords={one,three,five}}
\lstset{backgroundcolor=\color{gray},frame=single,emph={EMPTY},emphstyle=\color{white}, showstringspaces=false,commentstyle=\itshape, morekeywords={printf}, tabsize=2, fontadjust=true,
basicstyle=\footnotesize}
%\lstset{framexleftmargin=5mm, frame=shadowbox, rulesepcolor=\color{black}, backgroundcolor=\color[gray]{.9}}
\lstset{backgroundcolor=\color[gray]{.8}}

\usepackage{float}
\usepackage{rotating}
\usepackage{subfigure}
\usepackage{pdflscape}


\usepackage{pstricks}


\definecolor{code-color}{cmyk}{0, 0, 0, 0.1}

\usepackage{amsmath,amssymb,latexsym,array, bm, times}

\pagestyle{myheadings}
\markboth{Sistemas Embebidos}{Carlos Iv�n Camargo Bare�o}

\usepackage{longtable}
\parindent 1cm
\parskip 0.2cm
\topmargin 0.2cm
\oddsidemargin 1cm
\evensidemargin 0.5cm
\textwidth 15cm
\textheight 21cm


\begin{document}
\bibliographystyle{unsrt}


\parindent=1cm            
     \input title
\tableofcontents
%      \input intro
%      \input technology_transfer
%      \input description
     \input embedded
%     \input commons
%     \input education
%     \input platform

\bibliography{biblio}

\end{document} 




presentaci�n:

Tabla de contenido:

Cual es el problema , como se resuelve y contribuci�n.

Introducci�n:

El problema es:

Atraso existente en Colombia en dise�o digital; 
  - Inversi�n en I+D
  - N�mero de facultades de electr�nica, 
  - Porque fall� la transferencia.


Despu�s expicar el problema


Y despues decir que se hizo para resolver el problema.



!!!30 minutos !!!!  15 diapositivas



