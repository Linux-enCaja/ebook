\subsection{Adaptaci�n}
\begin{frame}
\frametitle{}

\begin{figure}
\begin{tikzpicture}[node distance=0.5cm, auto,>=latex', thick]
\scriptsize
    % We need to set at bounding box first. Otherwise the diagram
    % will change position for each frame.
    \path[use as bounding box] (-1.5,0) rectangle (12,-2);

    % TT methodology     
    \node [phase]                        (monitoreo)     {Vigilancia};
    \node [phase, below of=monitoreo]    (choice)        {Elecci�n};
    \node [phase, below of=choice]       (acquisition)   {Adquisici�n};
    \node [phase2,below of=acquisition]  (adaptation)    {Adaptaci�n};
    \node [phase, below of=adaptation]   (absortion)     {Absorci�n};
    \node [phase, below of=absortion]    (aplication)    {Aplicaci�n};
    \node [phase, below of=aplication]   (difusion)      {Difusi�n};

    %%%%%%%%%%%%%%%%%%%%%%%%%%%%%%%%%%%%%%%%%%%%&
    %            Adaptaci�n
    %%%%%%%%%%%%%%%%%%%%%%%%%%%%%%%%%%%%%%%%%%%%&

    \onslide<1> \node [ph_explain, right=1cm of adaptation.east] (exp_adaptation)    
    {
    \begin{center} \textbf{Adaptaci�n} \end{center}
    \begin{itemize}
     \item Se presenta cuando la sociedad encuentra posible y deseable realizar cambios para involucrar usos particulares de la tecnolog�a.
     \item Metodolog�a para el estudio gradual de la tecnolog�a
       \begin{itemize}
        \scriptsize
        \item Adquisici�n de un dispositivo comercial. 
        \item Aplicar ingenier�a inversa para identificar su arquitectura y forma de programaci�n.
        \item Generaci�n de aplicaciones similares a la original. 
        \item Dise�o y construcci�n local.
        \item Transmisi�n de conocimientos a la academia y a la industria.
        \item Documentaci�n del proceso a todo sector de la sociedad.
       \end{itemize}
    \end{itemize}
    };

    
\end{tikzpicture}
\end{figure}

\end{frame}