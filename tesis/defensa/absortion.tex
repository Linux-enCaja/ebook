\subsection{Absorci�n}
\begin{frame}
\frametitle{}

\begin{figure}
\begin{tikzpicture}[node distance=0.5cm, auto,>=latex', thick]
\scriptsize
    % We need to set at bounding box first. Otherwise the diagram
    % will change position for each frame.
    \path[use as bounding box] (-1.5,0) rectangle (12,-2);

    % TT methodology     
    \node [phase]                        (monitoreo)     {Vigilancia};
    \node [phase, below of=monitoreo]    (choice)        {Elecci�n};
    \node [phase, below of=choice]       (acquisition)   {Adquisici�n};
    \node [phase, below of=acquisition]  (adaptation)    {Adaptaci�n};
    \node [phase2,below of=adaptation]   (absortion)     {Absorci�n};
    \node [phase, below of=absortion]    (aplication)    {Aplicaci�n};
    \node [phase, below of=aplication]   (difusion)      {Difusi�n};

    %%%%%%%%%%%%%%%%%%%%%%%%%%%%%%%%%%%%%%%%%%%%&
    %            Absorci�n
    %%%%%%%%%%%%%%%%%%%%%%%%%%%%%%%%%%%%%%%%%%%%&
    \onslide<1> \node [ph_explain, right=1cm of adaptation.east] (exp_absortion)     
    {
      \begin{center} \textbf{Absorci�n} \end{center}
    \begin{itemize}
     \item La absorci�n es la capacidad del receptor para absorber tecnolog�a de un sector y la asimilaci�n es la capacidad de asimilar (analizar, procesar, interpretar y entender) y utilizarla en otro sector 
     \item Se deben generar dos tipos de habilidades para soportar la tecnolog�a:
       \begin{itemize}
        \scriptsize
        \item T�cnicas: hardware, sistemas operativos, redes, tecnolog�as de la comunicaci�n, aplicaciones SW.
        \item Humanas: Habilidades y conocimientos necesarios para desarrollar, mantener, manipular, adaptar al entorno local y futuro desarrollo.
       \end{itemize}

     \item  Mecanismos de aprendizaje para operar y cambiar la nueva tecnolog�a; 
      \begin{itemize}
        \scriptsize
        \item Banco de proyectos que pueden ser utilizados como base de futuros desarrollos.
        \item Cursos para la ense�anza de metodolog�as de dise�o y procesos de fabricaci�n.
      \end{itemize}
     \item  Metodolog�as de dise�o y procesos de fabricaci�n para generaci�n de productos propios.
    \end{itemize}
    };


    
\end{tikzpicture}
\end{figure}

\end{frame}