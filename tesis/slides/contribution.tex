\section{Aportes}
\begin{frame}
\frametitle{Aportes}

\begin{figure}
\begin{tikzpicture}[node distance=0.5cm, auto,>=latex', thick]
\scriptsize
    % We need to set at bounding box first. Otherwise the diagram
    % will change position for each frame.
    \path[use as bounding box] (-1.5,0) rectangle (12,-2);

    % TT methodology     
    \node [phase_TT]                        (monitoreo)     {Transferencia Tecnol�gica};
    \node [phase_PI, below of=monitoreo]    (choice)        {Propiedad Intelectual};
    \node [phase_DE, below of=choice]       (acquisition)   {Dise�o Electr�nico};
    \node [phase_EI, below of=acquisition]  (adaptation)    {Educaci�n en Ingenier�a};

    %%%%%%%%%%%%%%%%%%%%%%%%%%%%%%%%%%%%%%%%%%%%&
    %            Absorci�n
    %%%%%%%%%%%%%%%%%%%%%%%%%%%%%%%%%%%%%%%%%%%%&
    \onslide<1> \node [goals_TT2, right=.5cm of adaptation.east] (exp_absortion)     
    {
      \begin{itemize}
      \item Aplicaci�n de una metodolog�a de TT abierta e informal.
        \begin{itemize}
        \scriptsize
         \item Generaci�n de productos novedosos.
         \item Proceso documentado y difundido.
         \item Transferencia de conocimientos.
        \end{itemize}
      \end{itemize}
    };

    \onslide<2> \node [goals_PI2, right=.5cm of adaptation.east] (exp_adaptation)    
    {
      \begin{itemize}
      \item Definici�n del concepto hardware copyleft.
      \item Creaci�n de una comunidad basada en el conocimiento como bien p�blico.
      \item Creaci�n de un recurso p�blico representado en el conocimiento necesario para dise�ar sistemas digitales.
      \item Creaci�n de un mecanismo de difusi�n p�bico.
      \end{itemize}
    };

    \onslide<3> \node [goals_DE2, right=.5cm of adaptation.east] (exp_adaptation)    
    {
      \begin{itemize}
      \item Apropiaci�n de conocimientos necesarios para el dise�o y fabricaci�n de sistemas digitales modernos.
      \item Formulaci�n y aplicaci�n de metodolog�as de ingener�a inversa.
      \item Creaci�n de plataformas copyleft hardware, generaci�n de software b�sico y aplicaciones.
      \item Contribuci�n al movimiento de software libre
        \begin{itemize}
         \scriptsize
         \item Drivers de dispositivos para el kernel de Linux.
         \item Software para verificaci�n basado en el puerto JTAG.
         \item Aplicaciones para programaci�n en sistema.
        \end{itemize}
      \item Utilizaci�n de software libre para el proceso completo de dise�o.
      \end{itemize}
     };
 
    \onslide<4> \node [goals_EI2, right=.5cm of adaptation.east] (exp_adaptation)    
    {
      \begin{itemize}
      \item Formulaci�n y aplicaci�n de un plan de estudios basado en la iniciativa CDIO
        \begin{itemize}
         \scriptsize
         \item Creaci�n de habilidades necesarias para generar nuevos productos comercializables.
         \item Transferir los conocimientos adquiridos en el proceso de transferencia tecnol�gica.
        \end{itemize}
      \item Formaci�n de personal durante 2 a�os.
      \item Actualizaci�n de los programas relacionados con el dise�o digital
      \end{itemize}
    };

 
\end{tikzpicture}
\end{figure}

\end{frame}