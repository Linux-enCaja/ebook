\documentclass[xcolor=svgnames]{beamer}
% \usetheme{JuanLesPins}
\usetheme{CambridgeUS}
\usecolortheme{crane}
\usepackage{fontenc}

\usepackage[english]{babel}
\usepackage[latin1]{inputenc}
\usepackage{times}
\usepackage[T1]{fontenc}
\usepackage{fontenc}
\usepackage{amssymb}
\usepackage{multicol}
\usepackage{multirow}

\usepackage{amsmath,amssymb}
\usepackage{tikz}
\usepackage{verbatim}
\usepackage{subfigure}
\usetikzlibrary{arrows,shapes,positioning,backgrounds}
\usetikzlibrary{patterns}


\usepackage{times}
\usepackage{colortbl}
\usepackage{color}

\usepackage{tocvsec2}
\setlength{\leftmargini}   {13pt} 
\setlength{\leftmarginii}  {13pt} 
\setlength{\leftmarginiii} {13pt} 


\title[TT en Dise�o de Sistemas Embebidos]{TRANSFERENCIA TECNOL�GICA Y DE CONOCIMIENTOS EN EL DISE�O DE SISTEMAS EMBEBIDOS}
% \author{Carlos Iv�n Camargo Bare�o\\
%         Director: Luis Fernando Ni�o (PhD)
%         }

\author[Carlos Camargo]{Carlos Iv�n Camargo Bare�o\inst{1}  \\ \and Director: Luis Fernando Ni�o\inst{2}}
\institute[UNAL]{

\inst{1}Departamento de Ingenier�a El�ctrica y Electr�nica \and
\inst{2}Departamento de Ingenier�a de Sistemas e Industrial}

\pgfdeclareimage[height=.5cm]{logo}{../images/logo_unal}
\logo{\pgfuseimage{logo}}


\usepackage{etoolbox}
\makeatletter
\patchcmd{\@makefntext}{\insertfootnotetext{#1}}{\insertfootnotetext{\tiny#1}}{}{}
\makeatother

\begin{document}
\setbeamertemplate{navigation symbols}{}
\setbeamertemplate{footline}[page number]{}



\tikzstyle{hide}        = [text width=2cm,  text centered, minimum height=0.5cm]
\tikzstyle{phase}       = [draw, thin, fill=blue!20,  text width=2cm,   text centered, minimum height=0.5cm]
\tikzstyle{phase2}      = [draw, thin, fill=red!40,   text width=2cm,   text centered, minimum height=0.5cm]
\tikzstyle{extern}      = [draw, thin, fill=red!20,   text width=1.5cm, text centered, minimum height=0.5cm]
\tikzstyle{industry}    = [draw, thin, fill=green!20, text width=6cm,   text centered, minimum height=0.5cm]
\tikzstyle{academy}     = [draw, thin, fill=brown!20, text width=6cm,   text centered, minimum height=0.5cm]
\tikzstyle{ph_explain}  = [draw, thin, fill=gray!20,  text width=9cm,   text centered, minimum height=0.5cm]
\tikzstyle{ph_explain2} = [draw, thin, fill=green!20, text width=9.2cm, text centered, minimum height=0.5cm]
\tikzstyle{difusion_es} = [draw, thin, fill=yellow!20,text width=8.5cm, text centered, minimum height=0.5cm]
\tikzstyle{goals_PI}    = [draw, thin, fill=red!20,   text width=7cm,   text justified, minimum height=0.5cm]
\tikzstyle{goals_TT}    = [draw, thin, fill=green!20, text width=7cm,   text justified, minimum height=0.5cm]
\tikzstyle{goals_DE}    = [draw, thin, fill=blue!20,  text width=7cm,   text justified, minimum height=0.5cm]
\tikzstyle{goals_EI}    = [draw, thin, fill=yellow!20,text width=7cm,   text justified, minimum height=0.5cm]



\tikzstyle{goals_PI2}    = [draw, thin, fill=red!20,   text width=8cm,   text justified, minimum height=0.5cm]
\tikzstyle{goals_TT2}    = [draw, thin, fill=green!20, text width=8cm,   text justified, minimum height=0.5cm]
\tikzstyle{goals_DE2}    = [draw, thin, fill=blue!20,  text width=8cm,   text justified, minimum height=0.5cm]
\tikzstyle{goals_EI2}    = [draw, thin, fill=yellow!20,text width=8cm,   text justified, minimum height=0.5cm]

\tikzstyle{phase_PI}     = [draw, thin, fill=red!20,   text width=3.3cm,   text centered, minimum height=0.5cm]
\tikzstyle{phase_TT}     = [draw, thin, fill=green!20, text width=3.3cm,   text centered, minimum height=0.5cm]
\tikzstyle{phase_DE}     = [draw, thin, fill=blue!20,  text width=3.3cm,   text centered, minimum height=0.5cm]
\tikzstyle{phase_EI}     = [draw, thin, fill=yellow!20,text width=3.3cm,   text centered, minimum height=0.5cm]


\tikzset{
  myarrow/.style     = {->, >= triangle 90, thick},
  myarrow1/.style    = {->, >= triangle 90, thick, brown, line width=0.05cm},
  myarrow2/.style    = {->, >= triangle 90, thick, green, line width=0.05cm},
  mylabel/.style     = {text width=7em},
  mylabel2/.style    = {text width=20em}
} 

% Fija el tama�o de la indentaci�n en los 3 niveles de un enumerate, itemize, list
\setlength{\leftmargini}   {13pt} 
\setlength{\leftmarginii}  {13pt} 
\setlength{\leftmarginiii} {13pt} 


\pgfdeclarelayer{background}
\pgfdeclarelayer{foreground}
\pgfsetlayers{background,main,foreground}




%----------------------------------------------------------------------INSERT TITLE-
\frame[c]{\titlepage}
\settocdepth{subsection}


\begin{frame}
 \begin{alertblock}{Comit� Evaluador}
   \begin{itemize}
    \item PhD. Jos� Edinson Aedo Cobo - Universidad de Antioquia.
    \item PhD. Jos� Ignacio Martinez Torre - Universidad Rey Juan Carlos.
    \item PhD. Alvaro Zerda Sarmiento Universidad Nacional de Colombia.
   \end{itemize}

  
 \end{alertblock}

\end{frame}



\frame[c]{\tableofcontents}

      \input intro                  %  Introducci�n planteamiento del problema
      \input TT_methodology                  
      \input monitoring
      \input choice	
      \input acquisition
      \input adaptation
      \input absortion
      \input aplication  
      \input davinci
      \input difusion
      \input contribution
      \input conclusions
      \input biblio                 %  |Bibliograf�a

\end{document}|




Preguntas de investigaci�n:

Encontrar una metodolog�a que permita transferir conocimiento �tiles a la sociedad receptora.



