\section{Conclusiones}
\begin{frame}

\frametitle{Aportes}
  \begin{alertblock}{Absorci�n tecnol�gica}
       \begin{itemize}
        \item <1-> Metodolog�a para la adquisici�n de conocimiento basada en ingenier�a inversa.
          \begin{itemize}
            \item Conocimiento sobre arquitectura y dise�o de sistemas digitales.
            \item T�cnicas de fabricaci�n de prototipos y a escala.
            \item Programaci�n de sistemas digitales.
          \end{itemize}
        \item<2-> Metodolog�a de dise�o basado en herramientas abiertas.
          \begin{itemize}
            \item Flujo de dise�o hardware y software.
            \item Dise�o de PCBs.
          \end{itemize}
        \item<3-> T�cnicas de fabricaci�n de sistemas digitales.
        \item<4-> Programa acad�mico ajustado al CDIO.
        \item<5-> Mecanismo de difusi�n de los conocimientos adquiridos.
        \item<6-> Creaci�n de un recurso p�blico basado en el software y hardware libres.
       \end{itemize}    
  \end{alertblock}

\end{frame}

\begin{frame}
\frametitle{Conclusiones}
  \begin{alertblock}{}
    \begin{itemize}
     \item<1-> Se desarroll� y aplic� una metodolog�a para la transferencia tecnol�gica y de conocimientos en el �rea de dise�o de sistemas embebidos, centrada en la educaci�n y entrenamiento como canal para la transferencia.
     \item<2-> Se generaron conocimientos que pueden usarse como punto de partida para el desarrollo de nuevos productos.
     \item<3-> Se generaron cambios en el m�todo de ense�anza, integrando los conocimientos adquiridos.
    \end{itemize}
  \end{alertblock}
\end{frame}



\begin{frame}
\frametitle{Conclusiones}
  \begin{alertblock}{}
    \begin{itemize}
     \item<1-> Los centros acad�micos deben trabajar en problemas del entorno local.
     \item<2-> Es posible que la academia motive la creaci�n de empresas de base tecnol�gica.
     \item<3-> Toda actividad financiada por el gobierno debe proporcionar los medios necesarios para que sea difundida a la sociedad. 
     \item<4-> La metodolog�a presentada puede ser utilizada como referencia en procesos de transferencia tecnol�gica.
    \end{itemize}
  \end{alertblock}
\end{frame}


\begin{frame}
\frametitle{Trabajo Futuro}
 \only<1>{
  \begin{alertblock}{Academia}
    \begin{itemize}
     \item Aplicaci�n en la nueva maestr�a de Ingenier�a Electr�nica (UNAL).
     \item Medici�n del impacto en el pregrado.
     \item Aplicaci�n en la ense�anza media.
     \item Difusi�n en centros de acad�micos.
    \end{itemize}
  \end{alertblock}
 }
 \only<2>{
  \begin{alertblock}{Industria}
    \begin{itemize}
     \item Difusi�n de los conocimientos adquiridos.
     \item Incorporaci�n de empresas a la comunidad.
     \item Cumplimiento de normas internacionales.
    \end{itemize}
  \end{alertblock}
 }
 \only<3>{
  \begin{alertblock}{Gobierno}
    \begin{itemize}
     \item Difusi�n de resultados a organismos gunernamentales.
     \item Realizaci�n de proyectos.
    \end{itemize}
  \end{alertblock}
 }

\end{frame}
