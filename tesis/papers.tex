\appendix
\chapter{Art�culos}

\section{Aplicaciones}
\subsection{En Hardware Evolutivo}
Las plataformas desarrolladas en este trabajo fueron utilizadas como una forma novedosa y econ�mica de implementar algoritmos de hardware evolutivo, la posibilidad de tener una FPGA y un procesador en la misma placa permiti� la implementaci�n de algoritmos gen�ticos de forma intr�nseca, es decir, verificando el nivel de ajuste de cada individuo a la soluci�n requerida dentro del mismo dispositivo. En esta �rea se generaron dos art�culos:

\begin{itemize}
 \item \textbf{Intrinsic Evolvable Hardware for Combinatorial Synthesis Based on SoC+FPGA and GPU Platforms} Ser� publicado en la conferencia \textit{Genetic and Evolutionary Computation Conference} GECCO 2011; clasificada 7ma entre 711 conferencias en inteligencia artificial (AI conference ranking) y es pubicado por ACM (Association for Computing Machinery \footnote{ACM, es la sociedad educativa y cient�fica  inform�tica m�s grande del mundo.}) 

 \item \textbf{Low Cost Platform for Evolvable-Based Boolean Synthesis} Publicado en el 2nd IEEE \textit{Latin American Symposium on Circuits and Systems} LASCAS 2011.
\end{itemize}




\subsection{Aplicaciones en Rob�tica}

\begin{itemize}
 \item \textbf{CONTROL DE SISTEMAS PARALELOS INSPIRADO EN LA NATURALEZA} 3th IEEE Colombian Workshop on Circuits and Systems.
 \item \textbf{PLATAFORMAS ABIERTAS HARDWARE/SOFTWARE PARA APLICACIONES EN ROBOTICA}, V Congreso Internacional de Ingenier�a Mec�nica y III de Ingenier�a Mecatr�nica.
\end{itemize}



\section{Sobre la Metodolog�a}
\begin{itemize}
 \item \textbf{M�todo de Ense�anza/Aprendizaje en Sistemas Embebidos Utilizando Hardware Copyleft}  A publicar en: IEEE Latin America Transactions

 \item \textbf{Hardware copyleft como Herramienta para la Ense�anza de Sistemas Embebidos} Congreso Argentino de Sistemas Embebidos CASE 2011, Buenos Aires Argentina ISBN 978-987-9374-69-6

 \item \textbf{Metodolog�a Para la Transferencia Tecnol�gica en la Industria Electr�nica Basada en Software Libre y Hardware Copyleft}. VI Congreso Internacional de la Red de Investigaci�n Y Docencia en Innovaci�n Tecnol�gica RIDIT ``Innovaci�n, Empresa Y Regi�n''.

 \item \textbf{ECBOT y ECB\_AT91 Plataformas Abiertas para el Dise�o de Sistemas Embebidos y Co-Dise�o HW/SW}, VIII Jornadas de Computaci�n Reconfigurable y Aplicaciones, Madrid Espa�a.

 \item \textbf{Metodolog�a Para la Transferencia Tecnol�gica en la Industria Electr�nica Basada en Software Libre y Hardware Copyleft} XVII Workshop de Iberchip 2011.

 \item \textbf{SIE: Plataforma Hardware copyleft para la Ense�anza de Sistemas Digitales} XVII Workshop de Iberchip 2011.

 \item \textbf{Metodolog�a para la Transferencia de Conocimientos en el Dise�o de Sistemas Digitales} A publicar en la reuni�n nacional ACOFI 2011.
\end{itemize}


