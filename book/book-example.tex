\documentclass[a5paper,10pt]{memoir}
\usepackage[german]{babel}
\usepackage[T1]{fontenc}
\usepackage{palatino}
\newfont{\pplbxviii}{pplbc scaled 1800}%
\newfont{\pplbcxii}{pplbc scaled 1200}% Palatino Bold + Small Caps 

\usepackage[svgnames]{xcolor}

% Chapter Style
\makechapterstyle{daleif3}{
%\renewcommand\chapternamenum{}
%\renewcommand\printchaptername{}
\renewcommand{\chaptitlefont}{%
\color[cmyk]{.3,1.0,.85,.25}%
% \scshape\large\centering}
\scshape\huge\centering}
\renewcommand{\chapnumfont}{%
\color[cmyk]{.3,1.0,.85,.25}
\scshape}
\renewcommand{\chapnamefont}{%
\color[cmyk]{.3,1.0,.85,.25}%
% \scshape\huge\centering}
\scshape\large\centering}
}

\usepackage{hyperref}
\usepackage{memhfixc} % Needed after hyperref ?

\chapterstyle{daleif3} 

\begin{document}

%%% --------------------------------------------------------
%%% titlepage
%%% --------------------------------------------------------
\title{\pplbxviii State of the Art\\\Large ---Memoir Chapter Style---}
\author{\pplbcxii Edico}
\date{}
\maketitle
\thispagestyle{empty}
\clearpage

%%% --------------------------------------------------------
\frontmatter
%%% --------------------------------------------------------
\tableofcontents


%%% --------------------------------------------------------
\mainmatter
%%% --------------------------------------------------------

\chapter{A chapter title}
\pagecolor{blue!5}
\thispagestyle{empty}
\newpage

\pagecolor{white}
Some text at the beginning of a chapter. And we add a lot of text to
make sure that it spans more than one line.
\par\fancybreak{$***$}\par

\chapter*{A non-numbered chapter title}
\pagecolor{yellow!10} %
\thispagestyle{empty}
\newpage

\pagecolor{white}
Some text at the beginning of a chapter. And we add a lot of text to
make sure that it spans more than one line.
\end{document} 
