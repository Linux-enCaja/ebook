

\section{Bus Wishbone}

\subsection{Se�ales principales}

\begin{itemize}
 \item \textit{ack_o}: La activaci�n de esta se�al indica la terminaci�n normal de un ciclo del bus. 
 \item \textit{addr_i}: Bus de direcciones. 
 \item \textit{cyc_i}: Esta se�al se activa que un ciclo de bus v�lido se encuentra en progreso.
 \item \textit{sel_i}: Estas se�ales indican cuando se coloca un dato v�lido en el bus \textit{dat_i} durante un ciclo de escritura, y cuando deber�an estar presentes en el bus \textit{dat_o} durante un ciclo de lectura. El n�mero de se�ales depende de la granularidad del puerto. El LM32 maneja una granularidad de 8 bits sobre un bus de 32 bits, por lo tanto existen 4 se�ales para seleccionar el byte deseado (\textit{sel_i(3:0)}). 
 \item \textit{stb_i}: Cuando se activa esta se�al se indica al esclavo que ha sido seleccionado. Un esclavo wishbone debe responder a las otras se�ales �nicamente cuando se activa esta se�al. El esclavo debe activar la se�al \textit{ack_o} como respuesta a la activaci�n de \textit{stb_i}.
 
 \item \textit{we_i}: Esta se�al indica la direcci�n del flujo de datos, en un ciclo de lectura tiene un nivel l�gico bajo y en escritura tiene un nivel l�gico alto. 

 \item \textit{dat_i}: Bus de datos de entrada.
 \item \textit{dat_o}: Bus de datos de salida.
\end{itemize}









Volatile keyword says the compiler that no optimiztion on the variable.
The volatile keyword acts as a data type qualifier.
The volatile qualifier alters the default behaviour of the variable and does not attempt to optimize the storage referenced by it.
- Martin Leslie
volatile means the storage is likely to change at anytime and be changed but something outside the control of the user program. This means that if you reference the variable, the program should always check the physical address (ie a mapped input fifo), and not use it in a cacheed way.





