

General-Purpose Registers
The LatticeMico32 features the following 32-bit registers:
By convention, register 0 (r0) must always hold the value 0, and this is
required for correct operation by both the LatticeMico32 assembler and
the C compiler. On power-up, the value of 0 in r0 is not hardwired, so you
must initialize it to load r0 with the 0 value.
Registers 1 through 28 are truly general purpose and can be used as the
source or destination register for any instruction. After reset, the values in
all of these registers are undefined.
Register 29 (ra) is used by the call instruction to save the return address
but is otherwise general purpose.
Register 30 (ea) is used to save the value of the Program Counter (PC)
when an exception occurs, so it should not be used by user-level
programs.
Register 31 (ba) saves the value of the Program Counter (PC) when a
breakpoint or watchpoint exception occurs, so it should not be used by
user-level programs.



After reset, the values in all of the above 32-bit registers are undefined. To
ensure that register 0 contains 0, the first instruction executed after reset
should be xor r0, r0, r0.
Table 3 lists the general-purpose registers and specifies their use by the C
compiler. In this table, the callee is the function called by the caller function.
Table 3: General-Purpose Registers
Register Name Function r0 Holds the value zero r1 General-purpose/argument 0/return value 0 Caller
r2 General-purpose/argument 1/return value 1 Caller
r3 General-purpose/argument 2 Caller
r4 General-purpose/argument 3 Caller
r5 General-purpose/argument 4 Caller
r6 General-purpose/argument 5 Caller
r7 General-purpose/argument 6 Caller
r8 General-purpose/argument 7 Caller
r9 General-purpose Caller
r10 General-purpose Caller
r11 General-purpose Callee
r12 General-purpose Callee
r13 General-purpose Callee
r14 General-purpose Callee
r15 General-purpose Callee
r16 General-purpose Callee
r17 General-purpose Callee
r18 General-purpose Callee
r19 General-purpose Callee
r20 General-purpose Callee
r21 General-purpose Callee
r22 General-purpose Callee
r23 General-purpose Callee
r24 General-purpose Callee
r25 General-purpose Callee
r26/gp General-purpose/global pointer Callee
r27/fp General-purpose/frame pointer Callee
r28/sp Stack pointer Callee
r29/ra General-purpose/return address Caller
r30/ea Exception address r31/ba Breakpoint address 


Control and Status Registers
Table 4 shows all of the names of the control and status registers (CSR),
whether the register can be read from or written to, and the index used when
accessing the register. Some of the registers are optional, depending on the
configuration of the processor (see ?Configuring the LatticeMico32 Processor?
on page 35). All signal levels are active high.
Table 4: Control and Status Registers
Name
Access Index
Description
No 
PC
Optional Program counter
IE R/W 0x0 Yes Interrupt enable
EID R ? No LM32 revision number. EID is not
            accessible via Assembly, since it does not
           have an index.
IM R/W 0x1 Yes Interrupt mask
IP R 0x2 Yes Interrupt pending
ICC W 0x3 Yes Instruction cache control
DCC W 0x4 Yes Data cache control
CC R 0x5 Yes Cycle counter
CFG R 0x6 No Configuration
EBA R/W 0x7 No Exception base address


PC ? Program Counter
The PC CSR is a 32-bit register that contains the address of the instruction
currently being executed. Because all instructions are four bytes wide, the two
least significant bits of the PC are always zero. After reset, the value of the PC
CSR is h00000000.




EID ? Exception ID
The EID CSR is a 3-bit value that indicates what has caused the program
execution to stop. The values and corresponding events are the following:
0 ? Reset
1 ? Breakpoint Exception
2 ? Instruction Bus Error
3 ? Watchpoint Exception
4 ? Data Bust Error
5 ? Divide-by-zero
6 ? Interrupt
7 ? System Call
Exceptions are events either inside or outside of the processor that cause a
change in the normal flow of program execution. The LatticeMico32 processor
can raise eight types of exceptions, as shown in Table 10. One clock must be
held in active high until an acknowledge (ACK_I) is accepted. The remaining
are internal and no ACK_I is generated. The exceptions are listed in a
decreasing order of priority, so if multiple exceptions occur simultaneously, the
exception with the highest priority is raised.
Table 10: Exceptions
Exception ID Condition
Reset 0 Raised when the processor?s reset pin is asserted.
Breakpoint 1 Raised when either a break instruction is executed or when
            a hardware breakpoint is triggered.
InstructionBusError 2 Raised when an instruction fetch fails, typically due to the
                     requested address being invalid.
Watchpoint 3 Raised when a hardware watchpoint is triggered.
DataBusError 4 Raised when a data access fails, typically because either
              the requested address is invalid or the type of access is not
             allowed.
DivideByZero 5 Raised when an attempt is made to divide by zero.

Interrupt 6 Raised when one of the processor?s interrupt pins is
             asserted, providing that the corresponding field in the
            interrupt mask (IM) CSR is set and the global interrupt
           enable flag, IE.IE, is set.
SystemCall 7 Raised when an scall instruction is executed.


Exception Handlers
When an exception occurs, the CPU branches to an address that is an offset
from either the EBA CSR or the DEBA CSR. The offset is calculated by
multiplying the exception ID by 32. Exception IDs are shown in Figure 10 on
page 20. Since all LatticeMico32 instructions are four bytes long, this means
each exception handler can be eight instructions long. If further instructions
are required, the handler can call a subroutine.
Whether the EBA or DEBA is used as the base address depends upon the
type of the exception that occurred and whether DC.RE is set. Having two
different base addresses for the exception table allows a debug monitor to
exist in a different memory from the main program code. For example, the
debug monitor may exist in an on-chip ROM, whereas the main program code
may be in a DDR or SRAM. The DC.RE flag allows either interrupts to run at
full speed when debugging or for the debugger to take complete control and
handle all exceptions.
When an exception occurs, the only state that is automatically saved by the
CPU is the PC, which is saved in either ea or ba, and the interrupt enable flag,
IE.IE, which is saved in either IE.EIE or IE.BIE. It is the responsibility of the
exception handler to save and restore any other registers that it uses, if it
returns to the previously executing code. This will likely be the case for
interrupt exception handlers but may not be so for handlers that typically
result in the termination of a program, such as divide by zero or bus error
handlers.
The following piece of code shows how the exception handlers can be
implemented. The nops are required to ensure that the next exception handler
is aligned at the correct address.
To ensure that this code is at the correct address, it is common practice to
place it in its own section. Place the following assembler directive at the start
of the code:


/* Exception handlers */
_reset_handler:
xor
r0, r0, r0
bi
_crt0
nop
nop
nop
nop
nop
nop
_breakpoint_handler:
sw
(sp+0), ra
calli
save_all
mvi
r1, SIGTRAP
calli
raise
bi
restore_all_and_bret
nop
nop
nop
_instruction_bus_error_handler:
sw
(sp+0), ra
calli
save_all
mvi
r1, SIGSEGV
calli
raise
bi
restore_all_and_eret
nop
nop
nop
_watchpoint_handler:
sw
(sp+0), ra
calli
save_all
mvi
r1, SIGTRAP
calli
raise
bi
restore_all_and_bret
nop
nop
nop
_data_bus_error_handler:
sw
(sp+0), ra
calli
save_all
mvi
r1, SIGSEGV
calli
raise
bi
restore_all_and_eret
nop
nop
nop

_divide_by_zero_handler:
sw
(sp+0), ra
calli
save_all
mvi
r1, SIGFPE
calli
raise
bi
restore_all_and_eret
nop
nop
nop
_interrupt_handler:
sw
(sp+0), ra
calli
save_all
mvi
r1, SIGINT
calli
raise
bi
restore_all_and_eret
nop
nop
nop
_system_call_handler:
sw
(sp+0), ra
calli
save_all
mv
r1, sp
calli
handle_scall
bi
restore_all_and_eret
nop
nop
nop
_save_all:
addi
sp, sp, -56
/* Save all caller save registers onto the stack */
sw
(sp+4), r1
sw
(sp+8), r2
sw
(sp+12), r3
sw
(sp+16), r4
sw
(sp+20), r5
sw
(sp+24), r6
sw
(sp+28), r7
sw
(sp+32), r8
sw
(sp+36), r9
sw
(sp+40), r10
sw
(sp+48), ea
sw
(sp+52), ba
/* ra needs to be moved from initial stack location */
lw
r1, (sp+56)
sw
(sp+44), r1
ret


/* Restore all registers and return from exception */
_restore_all_and_eret:
lw
r1, (sp+4)
lw
r2, (sp+8)
lw
r3, (sp+12)
lw
r4, (sp+16)
lw
r5, (sp+20)
lw
r6, (sp+24)
lw
r7, (sp+28)
lw
r8, (sp+32)
lw
r9, (sp+36)
lw
r10, (sp+40)
lw
ra, (sp+44)
lw
ea, (sp+48)
lw
ba, (sp+52)
addi
sp, sp, 56
eret
/* Restore all registers and return from breakpoint */
_restore_all_and_bret:
lw
r1, (sp+4)
lw
r2, (sp+8)
lw
r3, (sp+12)
lw
r4, (sp+16)
lw
r5, (sp+20)
lw
r6, (sp+24)
lw
r7, (sp+28)
lw
r8, (sp+32)
lw
r9, (sp+36)
lw
r10, (sp+40)
lw
ra, (sp+44)
lw
ea, (sp+48)
lw
ba, (sp+52)
addi
sp, sp, 56
bret








IE ? Interrupt Enable
The IE CSR contains a single-bit flag, IE, that determines whether interrupts
are enabled. This flag has priority over the IM CSR. In addition, there are two
bits, BIE and EIE, that are used to save the value of the IE field when either a
breakpoint or other exception occurs. Each interrupt is associated with a
mask bit (IE bit) index


IM ? Interrupt Mask
The IM CSR contains an enable bit for each of the 32 interrupts. Bit 0
corresponds to interrupt 0. In order for an interrupt to be raised, both an
enable bit in this register and the IE flag in the IE CSR must be set to 1. After
reset, the value of the IM CSR is h00000000.
IP ? Interrupt Pending
The IP CSR contains a pending bit for each of the 32 interrupts. A pending bit
is set when the corresponding interrupt request line is asserted low. Bit 0
corresponds to interrupt 0. Bits in the IP CSR can be cleared by writing a 1
with the wcsr instruction. W









\section{Bus Wishbone}

\subsection{Se�ales principales}

\begin{itemize}
 \item \textit{ack_o}: La activaci�n de esta se�al indica la terminaci�n normal de un ciclo del bus. 
 \item \textit{addr_i}: Bus de direcciones. 
 \item \textit{cyc_i}: Esta se�al se activa que un ciclo de bus v�lido se encuentra en progreso.
 \item \textit{sel_i}: Estas se�ales indican cuando se coloca un dato v�lido en el bus \textit{dat_i} durante un ciclo de escritura, y cuando deber�an estar presentes en el bus \textit{dat_o} durante un ciclo de lectura. El n�mero de se�ales depende de la granularidad del puerto. El LM32 maneja una granularidad de 8 bits sobre un bus de 32 bits, por lo tanto existen 4 se�ales para seleccionar el byte deseado (\textit{sel_i(3:0)}). 
 \item \textit{stb_i}: Cuando se activa esta se�al se indica al esclavo que ha sido seleccionado. Un esclavo wishbone debe responder a las otras se�ales �nicamente cuando se activa esta se�al. El esclavo debe activar la se�al \textit{ack_o} como respuesta a la activaci�n de \textit{stb_i}.
 
 \item \textit{we_i}: Esta se�al indica la direcci�n del flujo de datos, en un ciclo de lectura tiene un nivel l�gico bajo y en escritura tiene un nivel l�gico alto. 

 \item \textit{dat_i}: Bus de datos de entrada.
 \item \textit{dat_o}: Bus de datos de salida.
\end{itemize}









Volatile keyword says the compiler that no optimiztion on the variable.
The volatile keyword acts as a data type qualifier.
The volatile qualifier alters the default behaviour of the variable and does not attempt to optimize the storage referenced by it.
- Martin Leslie
volatile means the storage is likely to change at anytime and be changed but something outside the control of the user program. This means that if you reference the variable, the program should always check the physical address (ie a mapped input fifo), and not use it in a cacheed way.





